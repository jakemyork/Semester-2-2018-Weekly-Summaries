\documentclass[journal]{IEEEtran}
%\documentclass[conference]{IEEEtran}
%\documentclass[conference,compsoc]{IEEEtran}
%\documentclass{aiaa-pretty}
%\documentclass{report}
%\documentclass[10pt,conference,a4paper]{IEEEtran}

\usepackage[ampersand]{easylist}
\usepackage{mathtools}
\usepackage{dsfont}
\usepackage{scrextend}
\usepackage{graphicx}
\graphicspath{ {images/} }
\usepackage{listings}
\usepackage{color}
\usepackage{algorithm2e} %for pseudo code
\usepackage{textcomp}
%\usepackage{graphicx}
%\usepackage{draftwatermark}
%\SetWatermarkLightness{0.75}
%\SetWatermarkScale{.5}

\begin{document}

\title{DSP Weekly Summary}
{}
\maketitle

\section{\textbf{Week 1}}
\subsection{\textbf{Euler's Forward Rectangular Method}}
Find the difference equation for:
\[
    D(s) = \frac{U(s)}{E(s)} = K_0 \frac{s + a}{s + b}
\]
\[
    \rightarrow sU(s) + bU(s) = K_0 (sE(s) + aE(s)) 
\]
The corresponding differential equation is:
\[
    \dot{u} + bu[k] = K_0(\dot{e} + ae[k]
\]
The forward rectangular method gives:
\[
    \frac{u[k + 1] - u[k]}{T} + bu[k] = K_0(\frac{e[k + 1] - e[k]}{T} + ae[k]
\]
\[
    \rightarrow u[k + 1] = (1 + bT)u[k] + (K_0aT + K_0)e[k] + K_0 e[k + 1]
\]
\subsection{\textbf{System Stabilisation Refresher}}
To stabilise a system plant, $G(s)$, with a lead compensator of the form:
\[
    D(s) = \frac{s + a}{s + b}
\]
the closed-loop equation will be:
\[
    G_{cl}(s) = \frac{D(s)G(s)}{1 + D(s)G(s)}
\]
So the characteristic equation we need to solve is:
\[
    1 + D(s)G(s) = 0
\]
The zeroes of this formula must all be in the left-half plane to ensure it is stable. Alternatively, there is the PID controller which takes the form:
\[
    PID(s) = K_p + \frac{K_i}{s} + K_ds
\]
\section{\textbf{Week 2}}
\subsection{\textbf{Z-Transform}}
\[
    \mathcal{Z}\{f[k]\}(z) = \sum_{k = -\infty}^{\infty} f[k]z^{-k}
\]
\[
    \mathcal{Z}\{f[k + n]\} = z^n\mathcal{Z}\{f[k]\} - z^nf[0] - z^{n -1}f[1] - z^{n - 2}f[2] ... \{n > 0\}
\]
To link between the $z$ and Laplace transforms, we take the Laplace transform of $f_s(t)$:
\[
    \mathcal{L}\{f_s(t)\}(s) = \sum_{k = 0}^{\infty} f[kT](e^{Ts})^{-k}
\]
where $z = e^{Ts}$ (note: by this same token, $s = \frac{1}{t}ln(z)$). For the $z$ transform to have a stable result, the poles must be located INSIDE the unit circle:
\[
	s = a + jb \rightarrow z = e^{sT} = e^ae^{jb} \rightarrow e^a < 1
\]
If the system transfer function is stable, we can then use the Tustin transformation to be sure that the $z$-poles are also stable.
\subsection{\textbf{Trapezoidal Rule}}
The trapezoidal rule maps between the $s$ and $z$ domains:
\[
    s \rightarrow \frac{2}{T}\frac{z - 1}{z + 1} \quad z \rightarrow \frac{2 + sT}{2 - sT}
\]
\subsection{\textbf{Tustin (Bilinear) Transformation}}
This rule maps between the $j\omega$ and $z$ domains:
\[
    \frac{z - 1}{z + 1} \rightarrow j tan(\frac{j\omega T}{2})
\]
\subsection{\textbf{Final Value Theorem}}
If the poles of $(z - 1)F(z)$ are inside the unit circle, then:
\[
    \lim_{k\to\infty} f[k] = \lim_{z\to1} \frac{z - 1}{z}F(z)
\]
\subsection{\textbf{Comb Function}}
The comb function is used for sampling:
\[
    comb(t) = \sum_{k = -\infty}^{\infty} e^{j2\pi kt}
\]
\subsection{\textbf{Finite Geometric Series Refresher}}
\[
    \sum_{n = 0}^{N - 1} r^n = \frac{1 - r^N}{1 - r}
\]
Example use with the $z$ transform:
\[
    |a| < 1 \quad k \in \mathds{Z}
\]
\[
    f[k] =
    \begin{cases}
    \ a^k, &\quad k \geq 0\\
    \ 0, &\quad else\\
    \end{cases}
\]
Using the definition of the $z$ transform, we get:
\[
    \rightarrow \lim_{K\to\infty} \sum_{k = 0}^{K - 1} a^k z^{-k} = \lim_{K\to\infty} \sum_{k = 0}^{K - 1} (\frac{a}{z})^k
\]
\[
    = \lim_{K\to\infty} \frac{1 - (\frac{a}{z})^K}{1 - (\frac{a}{z})} = \frac{1 - 0}{1 - (\frac{a}{z})}
\]
\[
    = \frac{1}{1 - (\frac{a}{z})} = \frac{z}{z - a}
\]
This derivation is particularly important to the tutorial questions for this week.
\subsection{\textbf{Some Common Z-Transforms from the Table}}
\[
    \frac{1}{s} \rightarrow \frac{z}{z - 1}
\]
\[
    \frac{1}{s^2} \rightarrow T\frac{z}{(z - 1)^2}
\]
\[
    \frac{2}{s^3} \rightarrow T^2\frac{z(z + 1)}{(z - 1)^3}
\]
\[
    \frac{1}{s + a} \rightarrow \frac{z}{z - e^{-aT}}
\]
\[
    \frac{1}{(s + a)^2} \rightarrow T\frac{ze^{-aT}}{(z - e^{-aT})^2}
\]
\[
    \frac{1}{(s + a)(s + b)} \rightarrow \frac{z(e^{-aT} - e^{-bT})}{(b - a)(z - e^{-aT})(z - e^{-bT})}
\]
\section{\textbf{Week 3}}
\subsection{\textbf{Sampled Continuous Functions}}
The sampled expression for the continuous function, $r(t)$, is $r_s(t)$. The Laplace transform of this sampled function is given by:
\[
	R_s(s) = \sum_{k = -\infty}^{\infty} r(kT)e^{-kTs}
\]
Using the $z$-transform definition of $z = e^{Ts}$ (and $s = \frac{1}{t}ln(z)$), we get the $z$-transform of this function to be:
\[
	R(z) = \sum_{k = -\infty}^{\infty} r(kT)z^{-k}
\]
The key point here is that $R_s(s) \leftrightarrow R(z)$. You get the $z$-transform from the sampled signal.
\subsection{\textbf{Zero Order Hold Operation}}
The zero order hold function is effectively a block in a block diagram that we can use to hold the sampled function at a value for a sampled period $T$. The transfer function is given by:
\[
	ZOH(s) = \frac{1 - e^{-Ts}}{s}
\]
\subsection{\textbf{Open and Closed Loop Discrete Systems}}
The transfer function of an open loop discrete system is:
\[
	G(z) = (1 - z^{-1}) \mathcal{Z}\{\frac{G(s)}{s}\}
\]
Adding a controller and closing the loop represents the system as per its continuous counterpart:
\[
	G_{cl}(z) = \frac{C(z)G(z)}{1 + C(z)G(z)}
\]
\subsection{\textbf{Pitfalls Regarding Sampling of Signals}}
If the input to a system, $R(s)$, is sampled and then passed through a function block, $G(s)$, then the output is:
\[
	Y(s) = R_s(s)G(s)
\]
and the sampled output is:
\[
	Y_s(s) = R_s(s)G_s(s)
\]
When converted to the $z$ domain, the transform would be represented as:
\[
	Y(z) = R(z)G(z)
\]
However, if the unsampled input to a system, $R(s)$, is passed through a function block, $G(s)$, then the output is:
\[
	Y(s) = R(s)G(s)
\]
and the sampled output is:
\[
	Y_s(s) = [R(s)G(s)]_s
\]
When converted to the $z$ domain, the transform would be represented as:
\[
	Y(z) = RG(z)
\]
\subsection{\textbf{Loop Gain}}
The loop gain of a system is defined as any function blocks that are present in the system if the loop is open. A feedback controller ($H(s)$, for example) would not be part of this function. Loop gain is defined as:
\[
	L(z) = C(z)G(z)
\]
More generally, the loop gain is defined as:
\[
	L(z) = \frac{N(z)}{(z - 1)^nD(z)} \quad for n \geq 0
\]
where $N(z)$ and $D(z)$ are the numerator and denominator polynomials with no roots at unity. $L(z)$ has $n$ poles at unity. It is therefore called a type $n$ system.
\subsection{\textbf{Steady State Error}}
Applying the final value theorem above, the steady state error function of a system is defined as:
\[
	e(\infty) = \lim_{z\to1} \frac{(z - 1)R(z)}{z(1 + L(z))}
\]
\subsection{\textbf{Position, Velocity and Acceleration Error Constants}}
\[
	K_p = \lim_{z\to1} L(z)
\]
\[
	K_v = \lim_{z\to1} \frac{(z - 1)L(z)}{T}
\]
\[
	K_p = \lim_{z\to1} \frac{(z - 1)^2L(z)}{T^2}
\]
There is a table in Lecture 3c that shows the steady state error values in response to different inputs, which shows where these constants come in. To remove any steady state error of a system, add poles at $z = 1$ (note: the closed loop system should still remain stable).
\end{document}
