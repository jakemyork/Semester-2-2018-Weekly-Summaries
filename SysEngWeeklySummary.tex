\documentclass[journal]{IEEEtran}
%\documentclass[conference]{IEEEtran}
%\documentclass[conference,compsoc]{IEEEtran}
%\documentclass{aiaa-pretty}
%\documentclass{report}
%\documentclass[10pt,conference,a4paper]{IEEEtran}

\usepackage[ampersand]{easylist}
\usepackage{mathtools}
\usepackage{scrextend}
\usepackage{graphicx}
\graphicspath{ {images/} }
\usepackage{listings}
\usepackage{color}
\usepackage{algorithm2e} %for pseudo code
\usepackage{textcomp}
\usepackage{hyperref}
%\usepackage{graphicx}
%\usepackage{draftwatermark}
%\SetWatermarkLightness{0.75}
%\SetWatermarkScale{.5}
\hypersetup{colorlinks=true, urlcolor=blue}
\begin{document}

\title{Systems Engineering Weekly Summary}
{}
\maketitle

\section{Week 1}
\subsection{\textbf{Standish Chaos Report}}
The 1994 report surveyed IT executive managers from large, medium and small US companies with management information systems. They had 365 respondents that represented 8,380 software applications in the market. The report can be found here: \href{https://www.projectsmart.co.uk/white-papers/chaos-report.pdf}{Standish Chaos Report}. It surveyed the managers on projects and found the following:
\begin{itemize}
	\item successful projects (on time and on budget): \textbf{16.2\%}
	\item challenged projects (completed, but not on time, on budget or with reduced functionality compared to specification): \textbf{52.7\%}
	\item impaired projects (cancelled at some point): \textbf{31.1\%}
\end{itemize}
\subsubsection{Success factors}
The project success factors were found to be:
\begin{itemize}
	\item user involvement: \textbf{15.9\%}
	\item executive management support: \textbf{13.9\%}
	\item clear statement of requirements: \textbf{13.0\%}
	\item proper planning: \textbf{9.6\%}
	\item realistic expectations: \textbf{8.2\%}
	\item smaller project milestones: \textbf{7.7\%}
	\item competent staff: \textbf{7.2\%}
	\item ownership: \textbf{5.3\%}
	\item clear vision and objectives: \textbf{2.9\%}
	\item hard-working, focused staff: \textbf{2.4\%}
	\item other: \textbf{13.9\%}
\end{itemize}
\subsubsection{Challenged factors}
The project challenged factors were found to be:
\begin{itemize}
	\item lack of user input: \textbf{12.8\%}
	\item incomplete requirements and specifications: \textbf{12.3\%}
	\item changing requirements and specifications: \textbf{11.8\%}
	\item lack of executive support: \textbf{7.5\%}
	\item technology incompetence: \textbf{7.0\%}
	\item lack of resources: \textbf{6.4\%}
	\item unrealistic expectations: \textbf{5.9\%}
	\item unclear objectives: \textbf{5.3\%}
	\item unrealistic time frames: \textbf{4.3\%}
	\item new technology: \textbf{3.7\%}
	\item other: \textbf{23.0\%}
\end{itemize}
\subsubsection{Impaired factors}
The project impaired factors were found to be:
\begin{itemize}
	\item incomplete requirements: \textbf{13.1\%}
	\item lack of user involvement: \textbf{12.4\%}
	\item lack of resources: \textbf{10.6\%}
	\item unrealistic expectations: \textbf{9.9\%}
	\item lack of executive support: \textbf{9.3\%}
	\item changing requirements and specifications: \textbf{8.7\%}
	\item lack of planning: \textbf{8.1\%}
	\item didn't need it any longer: \textbf{7.5\%}
	\item lack of IT management: \textbf{6.2\%}
	\item technology illiteracy: \textbf{4.3\%}
	\item other: \textbf{9.9\%}
\end{itemize}
The report also speaks at length about four case studies: the California DMV, American Airlines and CONFIRM Car Rental, Hyatt Hotels and Banco Itamarati.
\subsection{\textbf{Systems Engineering Job Ads}}
We were required to be able to list relevant skills required by companies when they recruit systems engineers. Some of these were:
\begin{itemize}
	\item proficiency in company-relevant software
	\item ability to manage whole of life cycle of engineering projects
	\item ability to maintain technical documentation and conduct technical investigations
	\item have a developed professional network within the industry
	\item have teamwork, communication skills and professionalism
	\item previous experience
	\item ability to troubleshoot software problems
	\item have a customer focus
	\begin{itemize}
		\item NV1 clearance
		\item citizenship
		\item experience on Defence projects
	\end{itemize}
\end{itemize}
\subsection{\textbf{Lecture 0 - Course Overview}}
\subsubsection{System Factors}
Types of systems are defined on page 4 of the textbook. When a system is designed, there are a number of necessary factors to consider. Broadly, these can be grouped as the acronym, POSTED:
\begin{itemize}
	\item people
	\item organisation
	\item support
	\item training
	\item equipment $\rightarrow$ specification $\rightarrow$ engineering $\rightarrow$ product
	\item doctrine
\end{itemize}
\subsubsection{System Life Cycle}
The system life cycle is broken down into four phases:
\begin{itemize}
	\item pre-acquisition phase: idea for a system being generated as a result of business planning, including consideration of possible options, research and development
	\item acquisition phase: bringing the chosen system into service, including definition of business/stakeholder requirements and engagement of contractors
	\begin{itemize}
		\item conceptual design phase: production of a set of clearly defined requirements in logical terms - this results in several key documents:
		\begin{itemize}
			\item Business Needs and Requirements (BNR)
			\item Stakeholder Needs and Requirements (SNR)
			\item System Requirement Specification (SyRS)
			\item System Design Review (SDR)	
		\end{itemize}
		\item 
	\end{itemize}
	\item utilisation phase: the functional life of the system, including maintenance, modification and upgrades
	\item retirement phase: end of the life cycle of the system as it no longer meets operational requirements - the end of this life cycle could be the start of a new life cycle with a different business (service aircraft being used for scenic flights, for example)
\end{itemize}
\subsection{\textbf{Lecture 1 - Intro to Systems Engineering}}
\subsubsection{Broad Description of Systems}
This can be one of two ways:
	\begin{itemize}
		\item logical/functional - what the system will do, described with a narrative or scenarios in mind;
		\item physical - the technical specifications of system elements, how they look, dimensions etc.
	\end{itemize}
\subsubsection{Conceptual Design Overview}
This marks a formal transition from the stakeholder requirements specification (which is business-y) to a complete logical, physical description of the system. It ensures proper definition of system technical requirements and integrates the appropriate stakeholders in decision making. In this phase, we transition from BNR and SNR to a full SyRS developed by requirements engineers and eventually to the functional baseline for the system. After this, we transition to the system design review to ensure that both engineers and stakeholders are satisfied with the logical and physical concept of the system and how it will be designed. The SDR confirms BNR, SNR and SyRS formally.
\subsubsection{Preliminary Design Overview}
This part of the design phase takes the functional baseline and allocates the development of each subsystem to specific configuration items through means of an allocated baseline (system requirements being allocated to subsystems). Finally, this phase has a preliminary design review; again, this formalises all decisions made during this phase.
\subsubsection{Detailed Design and Development Overview}
During this phase, each subsystem (and components thereof) is developed in accordance with the ABL, the SyRS and the StRS. This results in both the product baseline and a critical design review.
\subsubsection{Construction and Production Overview}
After design and detailed verification and validation of the entire system, components are produced in accordance with the PBL. This ends with a formal qualification review, which formalises the customer accepting the system in its current state from the contractor as designed to specification. This is informed by acceptance test and evaluation.
\subsubsection{Utilisation/Retirement Overview}
During this phase, the system undergoes operational use, maintenance programs, and modifications and upgrades as deemed necessary by the customer. Ultimately, the system is retired as it is no longer viable, necessary or redundant.
\subsubsection{Types of Development Approach}
These can be loosely grouped into a few categories, with the most common being the first one:
\begin{itemize}
	\item waterfall
	\item incremental
	\item spiral
	\item evolutionary
\end{itemize}
These will be individually defined later. For now, we need to tailor our approach always to maximise value/viability and minimise risk.
\subsubsection{System Need-To-Know}
There are five things we need to know clearly about a system that is being designed:
\begin{itemize}
	\item what the system will do;
	\item how it does its job and how well it does it;
	\item under what conditions it operates;
	\item what systems it needs to integrate with;
	\item how can we be absolutely sure that it succeeds in these tasks?
\end{itemize}

\section{Week 2}
\subsection{\textbf{Lecture 2 - Conceptual Design}}
\subsubsection{Stakeholder Requirements Specification}
This document should contain:
\begin{itemize}
	\item likely applications for the system;
	\item major operational characteristics;
	\item operational or safety constraints;
	\item external systems/interfaces;
	\item operational and support environment;
	\item the support concept to be employed.
\end{itemize}
\subsubsection{System Requirements Review}
This may be conducted periodically through the conceptual design phase to verify and approve versions of system-level requirements. The goal of this review process is to monitor and approve requirements on the way to the initial FBL. It allows requirements analysis to continue to lower levels of the system hierarchy by validating higher levels, providing a firm baseline for lower level analysis to work from.

\section{Week 3}
\subsection{\textbf{Lecture 3 - Preliminary Design}}
\subsubsection{\textbf{Requirements Allocation}}
Requirements allocation refers to the allocation of specific design requirements to elements of the design. This requires expertise in the domain of the system in question to understand which subsystems can handle which requirements. It's important to remember at this point that we don't want scope creep and extra functionality that hasn't been requested. Both the contractor and customer need to be able to look horizontally and understand the impact that requirements of a subsystem may have on other subsystems. This is particularly relevant when changes are proposed. It's really important to remember that requirements need to inform design choices and not the other way around.
\subsubsection{\textbf{Configuration Items}}
Subsystems are broken into configuration items. These could be in the form of hardware vs software or in elements of a subsystem being bought by the subcontractor as COTS items. The configuration of each item is managed separately for design, development, documentation, construction, auditing and testing. Items can be identified as CIs due to:
\begin{itemize}
	\item complexity;
	\item interfaces;
	\item use/function;
	\item commonality;
	\item single supplier ownership;
	\item criticality;
	\item maintenance and documentation needs.
\end{itemize}
The CI decision process belongs to the contractor but may be influenced by the customer, who may have particular constraints on suppliers, requirements for documentation and intellectual property rights (access to software under the hood etc).
\subsubsection{Interface Selection}
Interfaces between relevant subsystems are identified at this stage of design. These determine successful operation of the system once integrated, but also place limitations and requirements on individual subsystems/CIs. These interfaces can be:
\begin{itemize}
	\item physical: pipes, wires, fibre;
	\item electronic: wired/wireless;
	\item software: I/O, data format, protocols;
	\item human-machine: layouts, displays, seating, controls;
	\item electrical: voltage levels/types, frequency, tolerances;
	\item environmental: vibration, acoustic, thermal, magnetic, radiation;
	\item hydraulic/pneumatic: flow rates, temperatures, pressures.
\end{itemize}
\subsubsection{Subsystem Design Choices}
There are three broad choices available to subsystem developers:
\begin{itemize}
	\item buy the subsystem from a supplier and integrate (COTS):
	\begin{itemize}
		\item (+) readily available;
		\item (+) cheaper;
		\item (+) reduce technical risk and maintenance liability;
		\item (-) may not be suitable/approaching outdated state;
		\item (-) could be quite immature/unproven;
		\item (-) may have less support/documentation;
		\item (-) may have undesired functionality.
	\end{itemize}
	\item buy a COTS subsystem and modify to meet needs (modified COTS):
	\begin{itemize}
		\item (+) same advantages as COTS;
		\item (-) maintenance/support may be void if modified;
		\item (-) effort to modify could outweigh benefits of COTS in the first place.
	\end{itemize}
	\item design and build specific to purpose:
	\begin{itemize}
		\item (+) should meet exact needs;
		\item (+) should be fully understood by own organisation;
		\item (-) may not eventuate;
		\item (-) significant effort;
		\item (-) maintenance and support is wholly own responsibility.
	\end{itemize}
\end{itemize}
\subsubsection{The Design Space}
The reason we conduct requirements analysis and develop particular subsystems (that may not be optimal) is to develop as close to an optimal system-level solution as possible. System-level requirements outweigh everything at the subsystem level and below. Design and selection of subsystems may be required to be evaluated a number of times before preliminary design is complete and this may be influenced significantly by the Standish Report reasons given above as well as long project times that can lead to technological developments that influence the system.
\subsubsection{Preliminary Design Review}
Once	complete, preliminary design results in a description of the preliminary architecture of hardware, software, personnel and so on organised in such a way as to satisfy the series of applicable requirements. The main deliverable is a description of all of the subsystems and how they interface with each other. Once the design is mature enough, it is reviewed before releasing the Allocated Baseline (ABL). This is only after the customer is sufficiently satisfied with the system design. The review process investigates each CI, ensuring that requirements have been appropriately allocated.
\section{Week 4}
\subsection{\textbf{Lecture 4 - Detailed Design and Development}}
\subsubsection{\textbf{Product Baseline}}
After preliminary design, detailed design and development finalises the design of specific components that make up each CI. The realisation and documentation of all of these individual components is the Product Baseline (PBL). The PBL is a detailed description of the products that meet the requirements allocated to them in the ABL.
\subsubsection{\textbf{Detailed Design Process}}
Sufficient detail is required in the PBL to facilitate construction/production. Once this has been completed (the `what'), analysis on production methods must be done (the `how'). The process:
\begin{itemize}
	\item is iterative;
	\item features review/feedback;
	\item involves integration of assemblies/components;
	\item may result in prototypes;
	\item involves test and evaluation.
\end{itemize}
At each stage of integration (components, assemblies, subsystems), evaluation will take place to ensure that requirements are being met and that elements are integrating effectively.
\subsubsection{\textbf{Prototypes}}
Prototypes may be required to verify the design in a final state. These combine and integrate all lower-level components.
\subsubsection{\textbf{Critical Design Review}}
This is a major review of detailed design and development. It also marks the final stage before construction/production. Things reviewed include:
\begin{itemize}
	\item software products;
	\item design drawings;
	\item materials and parts lists;
	\item analyses and other reports.
\end{itemize}
Aims include:
\begin{itemize}
	\item design evaluation;
	\item determination of readiness for production;
	\item determination of maturity of software;
	\item determination of design compatibility;
	\item establishment of the PBL.
\end{itemize}
\subsection{\textbf{Lecture 4 - Construction/Production}}
This phase could be very large and require well-supported infrastructure ... or not. Compare a single flight simulator production and a fleet of armoured vehicles. The requirements of this phase need to be considered early in the design of the overall system.
\subsubsection{\textbf{Production Issues to Address}}
The following production issues must be addressed early in system development:
\begin{itemize}
	\item material availability (lead time), ordering and handling;
	\item availability of skill sets and requirement to train production force;
	\item availability of tools, equipment and facilities;
	\item processing and process control;
	\item assembly, inspection and testing facilities/requirements;
	\item packaging, storage and handling.
\end{itemize}
\subsubsection{\textbf{Production Plan}}
The contents of the Production Plan include:
\begin{itemize}
	\item resources:
	\begin{itemize}
		\item plant size and type (need to consider lead time, expense and any specialised plant requirement);
		\item personnel resources (need to consider number and specialisation of personnel, including training delta).
	\end{itemize}
	\item production engineerng considerations:
	\begin{itemize}
		\item scheduling;
		\item manufacturing methods/processes;
		\item tooling and test equipment;
		\item facility requirements;
		\item automation.
	\end{itemize}
	\item materials and purchased parts:
	\begin{itemize}
		\item Bill of Materials (BOM);
		\item procurement of any COTS CIs;
		\item identification and mitigation of long lead times;
		\item inventory control.
	\end{itemize}
	\item management and logistics;
	\item other activities such as test and evaluation;
	\item configuration audits.
\end{itemize}
\subsubsection{\textbf{Functional Configuration Audit}}
This is used to verify and certify that a CI meets performance requirements as specified in the Development and Product Specifications. This may not be a full review - it can be done incrementally throughout the development of the CI. Sometimes this audit cannot be done due to some CIs needing other CIs to be functional and integrated to work effectively. FCA will usually precede PCA so as to avoid failing the FCA and invalidating the PCA.
\subsubsection{\textbf{Physical Configuration Audit}}
This audit confirms that the as-built CI matches product specifications, design drawings and technical (simulation) data. A PCA is conducted on the first production version of each CI.
\subsubsection{\textbf{Test Readiness Review}}
Testing is expensive and time-consuming, involving highly trained personnel and special facilities/equipment. Sometimes, TRR is required to ensure a system is ready to undergo testing. The idea is to avoid T\&E on CIs or a system that is not properly mature enough. The TRR usually reviews a range of documentation, including:
\begin{itemize}
	\item verification plans;
	\item formal and informal test results;
	\item supporting documentation;
	\item support, test and equipment facilities.
\end{itemize}
\subsubsection{\textbf{Formal Qualification Review}}
FQR may be required to verify that all the CIs meet functional requirements once integrated. FQR verifies specifications in SyRS, development specifications and interface requirement specifications.
\subsubsection{\textbf{Configuration Management}}
CM is the act of keeping track of what is being designed, current versions of parts, requests for changes and results of audits.

\section{Week 5}
\subsection{\textbf{Lecture 5 - Systems Engineering Management and Risk}}

\section{Week 6}
\subsection{\textbf{Lecture 6 - Project Management Introduction}}
\subsubsection{Process Groups}
The five process groups are:
\begin{itemize}
	\item initiate;
	\item plan;
	\item execute;
	\item monitor and control;
	\item close.
\end{itemize}
\subsubsection{Knowledge areas}
The ten knowledge areas are:
\begin{itemize}
	\item time;
	\item scope;
	\item cost;
	\item quality;
	\item risk;
	\item human resources;
	\item procurement;
	\item communications;
	\item stakeholders;
	\item integration.
\end{itemize}
\subsubsection{Capability levels}
The capability levels of an organisation are:
\begin{itemize}
	\item level 0 - incomplete;
	\item level 1 - performed:
	\begin{itemize}
		\item ad hoc processes;
		\item variable outcomes;
		\item inconsistency between projects;
		\item depends on the quality of the individual project team;
		\item project awareness permeates the organisation.
	\end{itemize}
	\item level 2 - managed;
	\item level 3 - defined:
	\begin{itemize}
		\item mature processes;
		\item consistency between projects;
		\item project oriented systems;
		\item project awareness permeates the organisation.
	\end{itemize}
	\item level 4 - quantitatively managed;
	\item level 5 - optimised:
	\begin{itemize}
		\item continually improved processes;
		\item measured performance;
		\item agile and adaptive systems;
		\item empowered people.
	\end{itemize}
\end{itemize}
\subsubsection{Project initiation stage}
At this point, the project charter is written up by the company executive(s) or board and issued to the project manager. The project management plan (PMP) is also completed and gives the project team a document hierarchy to work to. At this point, the project team work to identify the scope of the project and collect requirements.
\section{Week 7}
\subsection{\textbf{Lecture 7 - Scope Management}}
\subsubsection{Steps of Scope Management}
Scope management steps:
\begin{itemize}
	\item plan scope management;
	\item collect requirements - identify scope;
	\item define scope;
	\item create work breakdown structure (WBS);
	\item validate scope;
	\item control scope.
\end{itemize}
\subsubsection{Inputs to Scope Management}
Inputs to this phase are:
\begin{itemize}
	\item project charter;
	\item PMP;
	\item enterprise environmental factors;
	\item organisational process assets;
	\item project documents:
	\begin{itemize}
		\item assumption log;
		\item lessons learned register;
		\item stakeholder register.
	\end{itemize}
	\item business case;
	\item agreements.
\end{itemize}
\subsubsection{Outputs from Scope Management}
Outputs from this phase are:
\begin{itemize}
	\item scope management plan;
	\item requirements management plan;
	\item requirements traceability matrix;
	\item create work breakdown structure (WBS);
	\item project scope statement;
	\item project documents updates:
	\begin{itemize}
		\item assumptions log;
		\item requirements traceability matrix;
		\item stakeholder register.		
	\end{itemize}
\end{itemize}
\subsubsection{Work Breakdown Structure}
Each level of the WBS covers 100\% of the project. Elements are outcome/deliverable focused, not action focused. Elements can be grouped according to things like phasing, performance groups or `like' system elements.
\subsubsection{Scope Management Processes}
\begin{itemize}
	\item planning process group:
	\begin{itemize}
		\item plan scope management;
		\item collect requirements;
		\item define scope;
		\item create WBS.		
	\end{itemize}
	\item monitoring and controlling process group:
	\begin{itemize}
		\item validate scope;
		\item control scope.		
	\end{itemize}
\end{itemize}
\section{Week 8}
	\begin{figure}[h]
		\hfill\includegraphics[width=.49\textwidth]{PMProcessGroups.jpg}\hspace*{\fill}
	\end{figure}
\subsection{\textbf{Lecture 8 - Risk Management}}
\subsubsection{Risk}
Risk = probability (0 to 1) * impact (+ or -). Negative impact is bad. Positive impact is good. Unlike a risk, a problem is something that \textbf{has} occurred.
\subsubsection{Project Risk Management Processes}
This process occurs after the propject charter has been written up. The project risk management processes are:
\begin{itemize}
	\item plan risk management, considering:
	\begin{itemize}
		\item nature/types of causes/consequences and how to measure them;
		\item how likelihood will be defined;
		\item timeframes of likelihoods/consequences;
		\item how levels of risk will be determined;
		\item tolerance thresholds.
	\end{itemize}
	\item identify risks;
	\item qualitative risk analysis;
	\item quantitative risk analysis;
	\item plan risk response;
	\item implement risk response plan;
	\item monitor risks.
\end{itemize}
The key output of these processes is the risk register.
\subsubsection{Risk Response}
Risk response is the process of developing options, selecting strategies, and agreeing on actions to address project risk exposure (enhance opportunities and to reduce threats) as well as treat project risks. This should be:
\begin{itemize}
	\item appropriate for the risk significance;
	\item cost effective;
	\item owned by a responsible agent;
	\item realistic.
\end{itemize}
A risk response plan documents how chosen treatment options will be implemented. Risk response can be broken into the following categories:
\begin{itemize}
	\item avoid (scheduling, strategy changes, scope adjustment, consultation);
	\item transfer (insurance, warranties);
	\item mitigate/reduce (increase testing, safer supplier options);
	\item accept (only when other strategies are non-viable or have led to risk becoming acceptably low).
\end{itemize}
\section{Week 9}
\subsection{\textbf{Lecture 9 - Time, Schedule and Cost Management}}
\subsubsection{Analogous Estimating}
This method of estimation uses historical data on similar projects/activities to make an estimate. Similarity is judged on type, size, complexity, physical characteristics etc. Expert judgement is needed to make the comparison. Take the Collins class submarines as an example of this being completely retarded.
\subsubsection{Parametric/Stochastic Estimating}
This method of estimation uses patterns that can be analysed statistically but not necessarily predicted accurately. Sometimes uses historical data and project parameters to estimate cost.
\subsubsection{Bottom-up/Engineering Estimating}
Here, estimates are done for each work package (lowest WBS level). The estimate and basis-of-estimate are recorded in the WBS dictionary. This is the only way to achieve a definitive estimate for the performance management baseline.
\subsubsection{Project Budget}
The budget is the amount allocated to the project to achieve its objectives (the funding requirement). It must cover the aggregated cost estimates but includes reserves:
\begin{itemize}
	\item contingency reserve:
	\begin{itemize}
		\item inside the cost baseline (project manager responsibility);
		\item to treat identified risks;
		\item includes allocated (assigned to work packages) and unallocated (held by project manager).
	\end{itemize}
	\item management reserve:
	\begin{itemize}
		\item kept outside the cost baseline (sponsor responsibility);
		\item for unforeseen changes in project scope;
		\item a formal change to the cost baseline is recorded as the reserve is consumed.
	\end{itemize}
\end{itemize}
\subsubsection{Critical Path}
This method estimates project duration and the flexibility of the schedule. Calculates early start, early finish, late start and late finish for each work package. \textbf{The critical path is the longest sequence of work packages}. Work packages on the critical path have no float. 
	\begin{figure}[h]
		\hfill\includegraphics[width=.49\textwidth]{criticalPath.jpg}\hspace*{\fill}
	\end{figure}
\subsubsection{Resource Levelling and Smoothing}
This keeps the allocation of each resource type below a given constraint or used in the most efficient way.
\subsubsection{Schedule Compression}
Shortening the schedule without changing the scope of the project:
\begin{itemize}
	\item crashing:
	\begin{itemize}
		\item add resources to critical work packages to shorten duration;
		\item often increases project costs;
		\item often increases risks;
		\item for example, hiring three extra programmers for a 1-week work package.
	\end{itemize}
	\item fast tracking:
	\begin{itemize}
		\item start a work package ahead of schedule;
		\item may increase risks and create rework;
		\item for example, start flight testing before hovering has passed integration testing.
	\end{itemize}
\end{itemize}
\subsubsection{Earned Value Management}
This approach shows deviations in project performance by combining measures of cost, schedule and resources. It uses three key metrics for each work package: \begin{itemize}
	\item planned value:
	\begin{itemize}
		\item work packages carry costs into time space;
		\item the aggregation of all work packages is the performance management baseline;
		\item s-curve in the time-phase budget.
	\end{itemize}
	\item actual costs:
	\begin{itemize}
		\item captured as they occur and posted to work packages.
	\end{itemize}
	\item earned value:
	\begin{itemize}
		\item each work package is assigned a technique that best represents real progress;
		\item periodic reporting of progress.
	\end{itemize}
\end{itemize}
The three measures are then compared to measure variance (negative variance is bad).
\section{Week 10}
\subsection{\textbf{Lecture 10 - Stakeholder and Communications Management}}
\subsubsection{Stakeholder Management Steps}
\begin{itemize}
	\item identify stakeholders (roles, interests, level of knowledge/influence, expectations):
	\begin{itemize}
		\item classify into levels of power/interest (potentially through salience model);
		\item generate stakeholder register (identification, assessment, classification).
	\end{itemize}
	\item plan stakeholder management:
	\begin{itemize}
		\item engagement levels: unaware, resistant, neutral, supportive, leading;
		\item identify strategies/actions to promote stakeholder involvement.
	\end{itemize}
	\item manage stakeholder engagement:
	\begin{itemize}
		\item meet needs, address issues, foster involvement;
		\item keep communications open;
		\item shift support towards project and meet expectations;
		\item develop issue log to track stakeholder concerns.
	\end{itemize}
	\item monitor stakeholder engagement (as best you can):
	\begin{itemize}
		\item adjust plans for engagement;
		\item apply communication and interpersonal skills.
	\end{itemize}
\end{itemize}
\subsubsection{Communication Management Steps}
\begin{itemize}
	\item plan communication management:
	\begin{itemize}
		\item develop an approach that meets stakeholder information needs;
		\item apply relevant and available communications assets
		\item develop communications management plan.
	\end{itemize}
	\item manage communications:
	\begin{itemize}
		\item collection, distribution, storage, retrieval, management, monitoring and disposition of information;
		\item must receive response - not just broadcasting.
	\end{itemize}
	\item control communications:
	\begin{itemize}
		\item ensuring information needs are met at all times.
	\end{itemize}
\end{itemize}
\subsection{\textbf{Lecture 11 - Development Models}}
\subsubsection{Waterfall Model}
Top-down flow through systems engineering process. Business value is delivered in one hit. Suitable when:
\begin{itemize}
	\item requirements are completely known;
	\item nature of requirement is stable;
	\item no unresolved, high-risk components;
	\item requirements are compatible with stakeholder expectations;
	\item implementation architecture is well understood;
	\item timeline allows for sequential development.
\end{itemize}
\subsubsection{Incremental Model}
Phased development of a known set of requirements (multi-waterfall). Business value is delivered in multiple hits. Suitable when:
\begin{itemize}
	\item partial capability builds can be developed/maintained;
	\item partial core capability is desired as early as possible;
	\item funding is split into increments.
\end{itemize}
\subsubsection{Evolutionary Model}
Refinement and discovery of requirements through phases. Business value is delivered progressively, with backtracking possible. Suitable when:
\begin{itemize}
	\item partial capability builds can be developed/maintained;
	\item usage of the system is expected to revise or discover requirements.
\end{itemize}
\subsubsection{Spiral Model}
Risk-driven process model generator for software projects. Iterates a set of elemental development processes to actively reduce risk, based on unique risk profile of a project. Spiral quadrants in order:
\begin{itemize}
	\item determine objectives (requirements planning);
	\item identify and resolve risks (develop prototypes);
	\item development and testing (verification/validation, detailed design, integration and implementation)
	\item plan the next iteration (test/development planning and then release).
\end{itemize}
\subsubsection{Spiral Model}
\begin{itemize}
	\item approach:
	\begin{itemize}
		\item working capability is the measure of progress;
		\item deliver capability early and continuously;
		\item welcome changing requirements;
		\item reduce documentation - focus on conversations;
		\item focus on architecture and technical excellence.
	\end{itemize}
	\item team:
	\begin{itemize}
		\item integrate business people with developers;
		\item let the dev team identify requirements and develop architecture;
		\item give the dev team authority;
		\item sustain and retain the dev team;
		\item adopt continuous team development.
	\end{itemize}
	\item challenges:
	\begin{itemize}
		\item active involvement from customer is mandatory;
		\item requires continuous planning and updating;
		\item formalised testing and acceptance criteria;
		\item more difficult for large teams;
		\item requires consistency across teams with respect to processes, skills and quality.
	\end{itemize}
\end{itemize}

\section{Week 11}
\subsection{Lecture 12 - Sustainment Management 1}
Operating and use costs always outweigh acquisition costs (totalling approximately 60 to 85\%). These factors include:
\begin{itemize}
	\item operations;
	\item human safety;
	\item reputation costs;
	\item professional responsibility and reputation;
	\item software and ICT systems;
	\item organisational change;
	\item organisation reputation;
	\item industry support;
	\item facilities;
	\item maintenance;
	\item spares;
	\item training;
	\item distribution;
	\item test and support equipment;
	\item technical data;
	\item environmental protection;
	\item retirement and disposal.
\end{itemize}
Sustainment includes:
\begin{itemize}
	\item supply support (procurement, positioning, distribution, storage, salvage);
	\item maintenance (operational, preventative, repair, corrective, predictive);
	\item transportation (packaging, handling, storage, distribution);
	\item upgrades and modification;
	\item personnel and training;
	\item habitability;
	\item usability;
	\item supportability (the degree to which a system can be supported - standardisation, interchangeability, accessibility, commonality);
	\item survivability (to continue to operate in a hostile environment);
	\item environment;
	\item WHS;
	\item anti-tamper provisions;
	\item configuration management (the process of establishing and maintaining consistency of a product's performance, functional and physical attributes, with its requirements, design and operational information through its life);
	\item system data management;
	\item critical information protection;
	\item ICT including national security systems;
	\item interoperability functions (the ability to work with other products or systems without any restricted access or implementation).
\end{itemize}
Topics/aspects of sustainment:
\begin{itemize}
	\item systems engineering;
	\item procurement project management;
	\item maintenance processes;
	\item ILS;
	\item product life cycle management;
	\item service management.
\end{itemize}
Major benefits arise from early integration of sustainment considerations:
\begin{itemize}
	\item operating phase costs greatly outweigh initial capital costs;
	\item total lifecycle costs should be a prime consideration for system acquisition;
	\item non-monetary risks must be reduced -- human death/injury, environmental damage, reputational risks and other legal risks;
	\item the sustainment system needs to be developed in parallel the mission system;
	\item this is difficult to achieve 360 degree processes because of ownership and procedural discontinuities.
\end{itemize}
\section{Week 12}
\subsection{Lecture 13 - Sustainment Management 2}
Life Cycle Cost Analysis is used as per the following picture:
	\begin{figure}[h]
		\hfill\includegraphics[width=.49\textwidth]{LCCA.jpg}\hspace*{\fill}
	\end{figure}
The typical method of LCCA is:
\begin{itemize}
	\item develop the model/s:
	\begin{itemize}
		\item usually a discounted cash flow model;
		\item prompted by standardised lists and checklists.
	\end{itemize}
	\item conduct cost estimates for each alternative;
	\item normalise data across alternatives;
	\item validate the data;
	\item conduct sensitivity analysis;
	\item conduct risk analysis;
	\item comparatively assess and rank alternatives;
	\item report and present;
	\item advocate the preferred alternative.
\end{itemize}
Typical life cycle cost elements for the pre-acquisition phase, acquisition phase, utilisation phase and retirement phase are included in the final lecture slides (13-16).
LCCA tools often include (example is ACEIT):
\begin{itemize}
	\item cost elements lists;
	\item standardised spreadsheets;
	\item statistical models;
	\item simulation tools;
	\item checklists;
	\item estimation techniques: analogy, parametric, engineering.
\end{itemize}
\subsection{Net Present Value Method}
This is a method of comparing the value of something in dollars compared to some possible future value:
\[
	NPV = \sum_{t=0}^N\frac{R_t}{(1+i)^t}
\]
where:
\begin{itemize}
	\item $N$ = number of periods modelled;
	\item $t$ = time period index;
	\item $R_t$ = net cash flow at time period $t$;
	\item $i$ = discount rate = average cost of capital (long term  government bod rate for government).
\end{itemize}
\subsection{Failure Rate}
The failure rate is given by:
\[
	\gamma = \frac{number of failures}{duration} = \frac{1}{MTBF}
\]
Reliability of a system is 1 - $\gamma$.
\subsection{Types of failure}
The following are the possible types of failure:
\begin{itemize}
	\item mission critical - cannot complete mission;
	\item safety critical - system safety is impacted;
	\item common mode - susceptible to a common cause;
	\item single point - failure of a single system element can cause system failure.
\end{itemize}
\subsection{Availability}
Availability is given by:
\[
	A = \frac{MTBF}{MTBF+MCMT}
\]
where MCMT is mean corrective maintenance time.
\subsection{TRF}
The formal model used by an organisation to control the quality of its materiel and to enact regulations. Comprises of:
\begin{itemize}
	\item governance - roles, responsibilities, authority, delegations, committees;
	\item processes;
	\item legislation;
	\item standards, regulation and other compliance requirements.
\end{itemize}
\subsection{Ways to Conduct Engineering Precedence}
The following are ways to conduct engineering precedence:
\begin{itemize}
	\item design for minimum risk - eliminate hazards where possible;
	\item incorporate safety devices - protective features;
	\item provide warning devices -  detect and annunciate;
	\item develop procedures and training - includes PPE;
	\item warnings and cautions.
\end{itemize}
\subsection{Failure Mode, Effects and Criticality Analysis}
This is a bottom-up, inductive analytical method used to identify failure modes
against the severity of their consequences. It is often mandated for certain industries and is often focussed on human safety.
\subsection{FMECA Method}
The following is the approach taken by the FMECA method:
\begin{itemize}
	\item define the system;
	\item define ground rules and assumptions in order to help drive the design;
	\item construct system block diagrams- eg functional block diagram, fault trees;
	\item identify failure modes (piece, part level or functional level);
	\item analyse failure effects and causes;
	\item classify the failure effects by severity;
	\item perform criticality calculations;
	\item rank failure mode criticality;
	\item determine critical items;
	\item identify the means of failure detection, isolation and compensation;
	\item perform maintainability analysis;
	\item document the results and specify design corrections;
	\item document uncorrectable design areas and identify the necessary special controls;
	\item make recommendations;
	\item follow up on corrective action implementation/effectiveness.
\end{itemize}
	\begin{figure}[h]
		\hfill\includegraphics[width=.49\textwidth]{whenDone.jpg}\hspace*{\fill}
	\end{figure}


\clearpage
\section{Definitions}
\subsubsection{Approach - Evolutionary}
Similar to the incremental approach; however this approach seeks to build upon previous builds and continue to evolve a system to satisfactory completion.
\subsubsection{Approach - Incremental}
For many reasons, it may be inadvisable to deliver a project in a single iteration (limited capability demand early, insufficient funds/time, incomplete understanding of system requirements, risk management). With a good understanding of the scope of the project, this approach allocates aspects to a series of increments to be delivered over time.
\subsubsection{Approach - Spiral}
Primarily a software model, this approach makes clear that a design development cannot be precisely determined prior to its full development and requires constant re-evaluation of the system at specific intervals. This allows for consideration given to changes in user perceptions, results of prototypes, technology advances, risk determinations, funding or other factors that may become relevant.
\subsubsection{Approach - Waterfall}
This approach relies on the development of a complete set of requirements at the system level, which influences and cascades into the subsystem requirements, which in turn, drive the requirements for the assemblies and components. At each transition, a baseline tends to be established.
\subsubsection{Construction}
The assembly and building of the system.
\subsubsection{Costs - Burdened}
Usually included with particular services/activities. Things like uniforms, employer-paid snacks, minor tools and equipment.
\subsubsection{Costs - Direct}
Costs that can be traced to a cost object (e.g. an output or process of the activity) with a high degree of accuracy.
\subsubsection{Costs - Indirect}
Costs that cannot be easily linked to a cost object, or where the costs of doing so outweigh the benefits. Indirect costs are apportioned to a cost object using an appropriate cost driver.
\subsubsection{Failure Criticality}
Likelihood of a critical failure ie mission failure including human safety.
\subsubsection{Failure Effect}
The consequence(s) a failure mode has on the operation, function, or status of an item.
\subsubsection{Failure Mode}
The way or manner in which an item fails - considered the `anti-function'.
\subsubsection{Failure Reliability Centered Maintenance (RCM)}
The function of the equipment is considered. Possible failure modes and their consequences are identified. Cost effective maintenance techniques that minimise failures are identified and adopted.
\subsubsection{Integrated Logistics Support (ILS)}
The functions that provides initial planning, funding and controls to ensure the user will receive a system that can be expeditiously and economically supported throughout its life cycle. Includes: maintenance, supply support, test and support equipment, personnel and training, facilities and equipment, packaging, handling, storage and transportation, technical data. ILS tasks: support concept, logistics support analysis, lifecycle cost analysis, inserting ILS considerations into system design and development.
\subsubsection{Life Cycle Cost Analysis}
Estimation and analysis techniques for the collection, analysis and presentation of Life Cycle Costing data to assist in decision making about required capabilities. Also called total cost of ownership.
\subsubsection{Product Life Cycle Management}
Also called asset management. Encompasses engineering change management. The system of strategic processes to:
\begin{itemize}
	\item reduce the cost of getting a product to market;
	\item efficiently scale production to meet market demand;
	\item extend the duration of the effective life of a system;
	\item continue maximum effectiveness through obsolescence;
	\item collaborate among supply chain team members to design, build and manage the system;
		\item share data across disparate systems for design, quality, manufacturing and operational systems;
	\item collect system data for communication and management of system issues;
	\item ensuring the system meets and retains compliance standards.
\end{itemize}
\subsubsection{Production}
The manufacturing and procurement effort needed to support construction.
\subsubsection{Project}
A Project is a temporary endeavour undertaken to create a unique product, service, or result.
\subsubsection{Project Charter}
This is a document that is written at the initiation phase of the project. It authorises the project and gives the project manager authority to commit resources. Details found in the charter are such as:
\begin{itemize}
	\item background and project purpose;
	\item objectives and success criteria;
	\item high-level requirements, scope, boundaries and constraints;
	\item key stakeholders;
	\item key deliverables;
	\item high-level risk assessment;
	\item high-level schedule requirements/deadlines;
	\item budget, authority and oversight processes;
	\item assignment of project manager;
	\item evidence of approval by sponsor.	
\end{itemize}
\subsubsection{Project Management}
Project Management is the application of knowledge, skills, tools and techniques to project activities to meet the project requirements.
\subsubsection{Project Management Plan}
A living document that “holds” the current status of all PM processes. Defines how the project will executed, monitored, controlled and closed.
\subsubsection{Risk}
An uncertain event or condition that, if it occurs, has a positive or negative effects on one or more project objectives such as scope, schedule, cost, and quality.
\subsubsection{Risk Management}
The systematic process of identifying, analysing, responding, and monitoring
project risk.
\subsubsection{Scope Creep}
Scope creep means changes, continuous or uncontrolled growth in the scope of a system's functionality, including the addition of undesired functionality as a byproduct of COTS CIs.
\subsubsection{Sustainment}
To keep something going or extending its duration. The provision of personnel, logistic and other support required to maintain and  prolong operations or combat until successful accomplishment or revision of the mission or of
the national objective.
\subsubsection{System}
ISO/IEC/IEEE 15288 defines a system as a combination of interacting elements organised to achieve one or more stated purposes. A system comprises internal elements with interconnections (subsystems) and an external boundary. Anything inside the boundary is defined as the system of interest. The mission of a system is to provide a solution to a business problem.
\subsubsection{Systems Engineering}
An interdisciplinary collaborative approach to derive, evolve, and verify a life-cycle balanced system solution which satisfies customer expectations and meets public acceptability.
\subsubsection{Traceability}
Forward	traceability	allows design decisions to be	traced from any requirement down	to a	lower level. Backward	 traceability means that any lower-level requirement is associated with at least one higher-level requirement.
\subsubsection{Validation}
This is the process of ensuring a system meets its operational purpose as dictated in the StRS.
\subsubsection{Verification}
This is the process of ensuring a system complies with the detailed specifications laid out in the SyRS, by meeting contractual requirements and performing adequately in an operational environment.
\subsubsection{Work Breakdown Structure}
A hierarchical decomposition of the total scope of work to be carried out by the project team to accomplish the project objectives and create the required deliverables.
\subsubsection{Work Breakdown Structure Dictionary}
Provides detailed information about each element of the WBS regarding scope, deliverables, scheduling, resources, costs etc.
\subsubsection{Work Package}
Deliverable or project work component at the lowest level of each
branch of the work breakdown structure. Can be definitively estimated and associated with a deliverable that can be objectively verified and accepted.
\section{Acronyms}
\begin{tabular}{ l l }
ABL & Allocated Baseline \\ 
AT\&E & Acceptance Test and Evaluation \\
BNR & Business Needs and Requirements \\
BOM & Bill of Materials \\
CDR & Critical Design Review \\
CI & Configuration Item/Critical Issue \\
CM & Configuration Management \\
CoC & Conditions of Contract \\
COI & Critical Operation Issue \\
COTS & Commercial Off The Shelf \\
DT\&E & Development Test and Evaluation \\
FBL & Functional Baseline \\
FCA & Functional Configuration Audit \\
FFBL & Functional Flow Block Diagram \\
FHA & Functional Hazard Analysis \\
FIC & Fundamental Inputs to Capability \\
FQR & Formal Qualification Review \\
FRACAS & Failure Reporting and Corrective Action System \\
ILS & Integrated Logistics Support \\
LCC & Life-Cycle Cost \\
MOE & Measures of Effectiveness \\
MOP & Measures of Performance \\
OT\&E & Operational Test and Evaluation \\
PBL & Product Baseline \\
PCA & Physical Configuration Audit \\
PDR & Preliminary Design Review \\
PM & Project Management \\
PMP & Project Management Plan \\
PRR & Product Readiness Review \\
RBS & Requirements Breakdown Structure \\
SDD & System Design Document \\
SDR & System Design Review \\
SE & Systems Engineering \\
SoR & Statement of Requirements \\
SoS & System of Systems \\
SoW & Statement of Work \\
SNR & Stakeholder Needs and Requirements \\
SRR & System Requirements Review \\
StRS & Stakeholder Requirement Specification \\
SyRS & System Requirement Specification \\
T\&E & Testing and Evaluation \\
TEMP & Test and Evaluation Master Plan \\
TPM & Technical Performance Measures \\
TRR & Test Readiness Review \\
WBS & Work Breakdown Structure \\
\end{tabular}
\end{document}
